\documentclass[conference]{IEEEtran}
\IEEEoverridecommandlockouts
% The preceding line is only needed to identify funding in the first footnote. If that is unneeded, please comment it out.
\usepackage{cite}
\usepackage{amsmath,amssymb,amsfonts}
\usepackage{algorithmic}
\usepackage{graphicx}
\usepackage{textcomp}
\usepackage{xcolor}
\def\BibTeX{{\rm B\kern-.05em{\sc i\kern-.025em b}\kern-.08em
    T\kern-.1667em\lower.7ex\hbox{E}\kern-.125emX}}
\begin{document}

\title{Study Aid Tool - Group 27 BACK PACS \\
{\large Aiden Burgess - abur970 - 600280511}
}
\maketitle
\begin{abstract}
This document is a model and instructions for \LaTeX.
This and the IEEEtran.cls file define the components of your paper [title, text, heads, etc.]. *CRITICAL: Do Not Use Symbols, Special Characters, Footnotes, 
or Math in Paper Title or Abstract.
\end{abstract}

\begin{IEEEkeywords}
component, formatting, style, styling, insert
\end{IEEEkeywords}

\section{Introduction}


\subsection{Motivation}
Students need to remove all distractions to study effectively. However, the process of setting up a relaxing environment for a session of focused work can be tedious and repetitive. Many students prefer to study with music in the background and with a nice wallpaper. When working on large problems or many problems, they like to break them down into smaller tasks in a todo list. It has become increasingly popular to use time-boxed techniques to maximise efficiency and focus, such as Pomodoro. Therefore, students may need to juggle many apps simultaneously, which can be distracting and annoying.

A unique feature that the team thought would be worthwhile to introduce is the ability to change the background and music based on the user's current mood.

\subsection{Goals}
The goals for this project were separated by priority into must-haves, should-haves, could-haves, and nice-to-haves. The list of proposed goals for the project can be found in Table 1.

As a summary, the main goals for the project was the mood-based functionality, which interacted with the background and music, and the todo list. These were chosen as the main goals because the unique feature of a mood.... The todo list is one of the most important components needed when a student is studying as it allows them to track their goals and progress for the study session.

With the same reasoning, white noise and a timer were should have functionality, as they aid a user's study.

\begin{table}[htbp]
\caption{Project Goals.}\label{tab1}
\resizebox{\columnwidth}{!}{\begin{tabular}{|p{18mm}|p{13mm}|p{45mm}|}
\cline{1-3} 
\textbf{\textit{Feature}}& \textbf{\textit{Time Estimate (Hours)}}& \textbf{\textit{Feature Description and Estimate Justification}} \\
\hline
\multicolumn{3}{|c|}{\textbf{Must-Haves}} \\
\hline
Mood based music player and recommendations & 20 & Get moods/preference from a mood slider and generate a playlist using the Spotify API that can be played in the browser. \\
\hline
Mood backgrounds & 12 & Generate some backgrounds based on the mood sliders/settings. Uses the splash API to retrieve backgrounds based on the search term. Create a slideshow with customisable interval. \\
\hline
Todo list & 10 & The ability to add items to a todo list which can help the user keep track of their progress.
Does not use third-party tools. \\
\hline
\multicolumn{3}{|c|}{\textbf{Should-Haves}} \\
\hline
White noise & 8 & A feature that will generate white noise for the user to aid in their study.
Simple white noise audio file that can play/pause. Reuse styling from the music recommender above. \\
\hline
Timer & 6 & A timer that can countdown and allow the user to keep track of time and perhaps implement study strategies like Pomodoro. \\
\hline
\multicolumn{3}{|c|}{\textbf{Could-Haves}} \\
\hline
Whiteboard & 8 & A whiteboard that the user can use to draw and save some of the images to help with their study. 
Use Third party components. \\
\hline
Study statistics & 15 & Statistics that summarise the user’s music time, study time, items added and completed. This can be used to help the student with seeing how much progress they are making. \\
\hline
User accounts & 20 & User accounts which remember the user’s preferences, with login details, and the ability to access preferences from multiple accounts.
Using a database over local storage. \\
\hline
\multicolumn{3}{|c|}{\textbf{Nice-To-Haves}} \\
\hline 
Music recommendations based on weather & 10 & Get the user location, and weather, then give a music recommendation using the Music Player already implemented. \\
\hline
Inspirational messages & 5 & Generating inspirational messages using API to display. \\
\hline
\end{tabular}}
\end{table}


\section{Related Work}
What has been done before? Compare it with your project.

BELOW IS ALL COPIED, NEED TO CHANGE AND MERGE WITH PRESENTATION

The preliminary market research revealed that no app exists with all the proposed app’s functionality. However, some apps/services exist which partially provide one of the features from our proposed feature-set. Below is a discussion on other apps which offer our top three features.

Spotify provides in-app music recommendations based on the user’s song history. Besides individual songs, Spotify recommends playlists which we will reproduce in our app. However, Spotify does not have any mood sliders based on which the current queue of songs. This is a feature which we will add to our app. 

The ‘mood backgrounds’ feature is something that no app currently offers. However, an app called ‘MoodTurn’ allows the user to choose a background category such as ‘beach’ or ‘nature’, then starts a background slideshow whilst playing soothing music associated with the pictures. We aim to keep the slideshow feature from this app, but instead of picture categories, we will have a way to change the pictures dynamically based on the user’s mood. 

Todoist is a comprehensive app that allows the user to create todo lists. It has a plethora of features, such as creating recurring events and delegating tasks to other people. While we will keep some basic features of this app, such as arranging the tasks date-wise, we do not aim to include the other complex features in our app.


\section{Design}
Software architecture (e.g. class diagram)? User interface (e.g. screen diagram)? Why this design?

\section{Implementation}
What have we implemented? 
Split into team contribution and individual?
Add some screenshots. Homepage (which I did)

\section{Testing}
How have we tested?
Frontend testing

Backend testing


\section{Methodology}
Management of your team in the project.

\subsection{Technologies}
Many technologies were used to co-ordinate and manage the team.
\subsubsection{Github}
Can add a screenshot of branches
Github is a VCS
\subsubsection{Jira}
Can add a screenshot of stats
Jira is a 
\subsubsection{Discord}
Can add a screenshot
Discord is a 
\subsubsection{Messenger}
Messenger is a 
\subsubsection{Zoom}
Zoom is a 
\subsection{Agile}
Can add a screenshot of burndown
The Agile methodology is...
One-week sprints, Jira helped track these. Sprints started on Monday, held a mid-sprint review on Thursdays to check if anyone has any significant blockers or interesting things to report.

Tickets were assigned 

Who created the issues



\section{Discussion}

\subsection{Progress on Goals}
Have the goals been achieved? Problems?
Our team implemented all of the must-haves, two of the should-haves, and one of the could-haves.

\subsection{Challenges}
One of the challenges we faced was that Sam was in Australia for the first 7 weeks of the project, so we had to hold our meetings online via Discord and Zoom

Another challenge was working with the spotify player api, as many teams have discussed previously, there are some frustrating parts of the spotify api, such as needing spotify premium to play music through spotify player api, and authentication protocols were proprietary and difficult to learn


\section{Conclusion}
Conclusions? Lessons? Future work?


\textbf{The class file is designed for, but not limited to, six authors.} A 
minimum of one author is required for all conference articles. Author names 
should be listed starting from left to right and then moving down to the 
next line. This is the author sequence that will be used in future citations 
and by indexing services. Names should not be listed in columns nor group by 
affiliation. Please keep your affiliations as succinct as possible (for 
example, do not differentiate among departments of the same organization).

\subsection{Identify the Headings}
Headings, or heads, are organizational devices that guide the reader through 
your paper. There are two types: component heads and text heads.

Component heads identify the different components of your paper and are not 
topically subordinate to each other. Examples include Acknowledgments and 
References and, for these, the correct style to use is ``Heading 5''. Use 
``figure caption'' for your Figure captions, and ``table head'' for your 
table title. Run-in heads, such as ``Abstract'', will require you to apply a 
style (in this case, italic) in addition to the style provided by the drop 
down menu to differentiate the head from the text.

Text heads organize the topics on a relational, hierarchical basis. For 
example, the paper title is the primary text head because all subsequent 
material relates and elaborates on this one topic. If there are two or more 
sub-topics, the next level head (uppercase Roman numerals) should be used 
and, conversely, if there are not at least two sub-topics, then no subheads 
should be introduced.

\subsection{Figures and Tables}
\paragraph{Positioning Figures and Tables} Place figures and tables at the top and 
bottom of columns. Avoid placing them in the middle of columns. Large 
figures and tables may span across both columns. Figure captions should be 
below the figures; table heads should appear above the tables. Insert 
figures and tables after they are cited in the text. Use the abbreviation 
``Fig.~\ref{fig}'', even at the beginning of a sentence.

\begin{figure}[htbp]
\centerline{\includegraphics{fig1.png}}
\caption{Example of a figure caption.}
\label{fig}
\end{figure}


Figure Labels: Use 8 point Times New Roman for Figure labels. Use words 
rather than symbols or abbreviations when writing Figure axis labels to 
avoid confusing the reader. As an example, write the quantity 
``Magnetization'', or ``Magnetization, M'', not just ``M''. If including 
units in the label, present them within parentheses. Do not label axes only 
with units. In the example, write ``Magnetization (A/m)'' or ``Magnetization 
\{A[m(1)]\}'', not just ``A/m''. Do not label axes with a ratio of 
quantities and units. For example, write ``Temperature (K)'', not 
``Temperature/K''.

\section*{Acknowledgment}

The preferred spelling of the word ``acknowledgment'' in America is without 
an ``e'' after the ``g''. Avoid the stilted expression ``one of us (R. B. 
G.) thanks $\ldots$''. Instead, try ``R. B. G. thanks$\ldots$''. Put sponsor 
acknowledgments in the unnumbered footnote on the first page.

\section*{References}

Please number citations consecutively within brackets \cite{b1}. The 
sentence punctuation follows the bracket \cite{b2}. Refer simply to the reference 
number, as in \cite{b3}---do not use ``Ref. \cite{b3}'' or ``reference \cite{b3}'' except at 
the beginning of a sentence: ``Reference \cite{b3} was the first $\ldots$''

Number footnotes separately in superscripts. Place the actual footnote at 
the bottom of the column in which it was cited. Do not put footnotes in the 
abstract or reference list. Use letters for table footnotes.

Unless there are six authors or more give all authors' names; do not use 
``et al.''. Papers that have not been published, even if they have been 
submitted for publication, should be cited as ``unpublished'' \cite{b4}. Papers 
that have been accepted for publication should be cited as ``in press'' \cite{b5}. 
Capitalize only the first word in a paper title, except for proper nouns and 
element symbols.

For papers published in translation journals, please give the English 
citation first, followed by the original foreign-language citation \cite{b6}.

\begin{thebibliography}{00}
\bibitem{b1} G. Eason, B. Noble, and I. N. Sneddon, ``On certain integrals of Lipschitz-Hankel type involving products of Bessel functions,'' Phil. Trans. Roy. Soc. London, vol. A247, pp. 529--551, April 1955.
\bibitem{b2} J. Clerk Maxwell, A Treatise on Electricity and Magnetism, 3rd ed., vol. 2. Oxford: Clarendon, 1892, pp.68--73.
\bibitem{b3} I. S. Jacobs and C. P. Bean, ``Fine particles, thin films and exchange anisotropy,'' in Magnetism, vol. III, G. T. Rado and H. Suhl, Eds. New York: Academic, 1963, pp. 271--350.
\bibitem{b4} K. Elissa, ``Title of paper if known,'' unpublished.
\bibitem{b5} R. Nicole, ``Title of paper with only first word capitalized,'' J. Name Stand. Abbrev., in press.
\bibitem{b6} Y. Yorozu, M. Hirano, K. Oka, and Y. Tagawa, ``Electron spectroscopy studies on magneto-optical media and plastic substrate interface,'' IEEE Transl. J. Magn. Japan, vol. 2, pp. 740--741, August 1987 [Digests 9th Annual Conf. Magnetics Japan, p. 301, 1982].
\bibitem{b7} M. Young, The Technical Writer's Handbook. Mill Valley, CA: University Science, 1989.
\end{thebibliography}
\vspace{12pt}
\color{red}
IEEE conference templates contain guidance text for composing and formatting conference papers. Please ensure that all template text is removed from your conference paper prior to submission to the conference. Failure to remove the template text from your paper may result in your paper not being published.

\end{document}
